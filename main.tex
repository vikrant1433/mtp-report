\documentclass[seminar]{report} \usepackage{longtable} \usepackage{tabu}
\usepackage{pifont} \usepackage{natbib} \usepackage{graphicx}
\usepackage{caption} \usepackage{subcaption}

% used for \textquotesingle
\usepackage{textcomp}

% used for degree
\usepackage{gensymb} \renewcommand\thesection{\arabic{section}}
\renewcommand{\bibname}{References}

\begin{document}

\begin{center}
  {\LARGE \textbf{GPS Analysis and Design Improvements \\[0.1in] in ``SafeStreet" }}\\[0.4in]
  {\large \textbf{M.Tech Thesis Report}}\\[0.2in]
  {\large \textit{Submitted in partial fulfillment of the requirements}\\
    for the degree of}\\[0.2in]
  {\large \textbf{Master of Technology\\
      in\\
      Computer Science and Engineering}}\\[0.2in]
  \large by\\[0.1in]
  {\Large Vikrant\\Roll No.: 153050031}\\[0.1in]

  \Large \textit{under the guidance of}\\[0.1in]
  {\Large Prof. Bhaskaran Raman}\\[0.3in]
  \begin{figure}[h]
    \centering
    \includegraphics[width=2in]{iitb-black}
  \end{figure}

  {\Large Department of Computer Science and Engineering\\[0.1in]
    Indian Institute of Technology Bombay\\[0.1in]
    October, 2016 }
\end{center}
\pagebreak

\begin{abstract}
\end{abstract}

\pagenumbering{roman}
\tableofcontents


\setcounter{page}{1} \pagenumbering{arabic}

\include{introduction/introduction}
% \section{GPS Analysis}
\subsection{How does GPS works ?}
Before we dig deep into the analysis GPS thing lets first understand how does GPS works?

As we know that GPS works on the principle of trilateration which means that if we
want to find out the location of a point in space and we know that it lies in three
planes then we can uniquely find out this point. Yes you guessed it right it the
point of intersection of all three planes. But in practice 

\begin{figure}
  \centering
  \begin{minipage}{.5\textwidth}
    \centering
    \includegraphics[width=.4\linewidth]{analysis/satellite}
    \captionof{A cell phone receiving signal from satellites in sky}
    
  \end{minipage}%
\end{figure}
\begin{figure}[h]
  \centering
  \includegraphics[width=2in]{analysis/trilateration}
  \caption{Trilateration}
  % \ref{trilateration}
\end{figure}


\section{GPS Analysis}
\subsection{How GPS works ?}
Before we dig deep into the GPS analysis lets first understand how GPS works?

\begin{figure}[h]
  \centering
  \includegraphics[width=2.5in]{analysis/satellite}
  \caption{A cell phone receiving signals from 4 satellites}
  \label{satellite}
\end{figure}
As we know that GPS works on the principle of trilateration which means that if
we want to find out the location of a point in space and we know
it\textquotesingle s distance from three points then we can uniquely identify
this point. Yes you guessed it right! it is the point of intersection of three
spheres with the radius equal to the distance from the point in question to
their respective centers.
\begin{figure}[h]
  \centering
  \includegraphics[width=2.5in]{analysis/trilateration}
  \caption{Trilateration}
  \label{trilateration}
\end{figure}

GPS receivers uses accurate time difference to calculate it\textquotesingle s
distance from satellites, which is equal to the time of arrival of GPS signal
minus the signal sending time. But it is not that simple to get the time in
nonoseconds range with our cell phones as it doesn't have atomic clock like
satellites have. Just to get an idea of why we need nanosecond accuracy in time
difference, an error of just 30 nanoseconds in time can lead to an error of 9
meters on Earth which is large enough to changed your pothole location from one
lane to another. To solve this need of high accuracy clocks in our cell phones,
cell phones uses 4 satellites to determine the exact time.

Now we have the basic understanding of GPS, lets try to understand some of the
GPS terminologies that we have used in this report.

% In this report we have done the analysis of GPS data obtained in our data
% collection phase.

\subsection{GPS Terminologies}
Lets first learn some of complex terminologies related to GPS.
% First is GPS latitude, it is same as geographic latitude that we already know
% about. Just to refresh our memory, latitude of a point on Earth's surface is
% the angle formed by the equitorial line passing through the center of Earth
% and the point joining the center of Earth. Latitude of north pole is 90\degree
% N and south pole is 90\degree S which is 90 and -90 repectively in decimal
% format.

% Second is GPS longitude, it is same as geographic longitude. Longitude of a
% point on Earth's surface is the angle formed by the plane passing through the
% Prime Meridian (which is Greenwich, England) containing North and South poles
% and the plane containing the point which also contains North and South poles.
% It is 0\degree for Greenwich, England and 0 to +180\degree for east side
% points and 0 to -180\degree for west side points.

\begin{enumerate}
\item GPS \textbf{accuracy}, This is a number that we get from our GPS device
  and it has a unit of meters. This is an important attribute for GPS location
  % and we will spend most of our time around it.
  The definition of accuracy according to \textit{developer.android.com} website
  is ``the radius of \textbf{68\%} confidence''. In other words, if you draw a
  circle centered at this location's latitude and longitude with a radius equal
  to the accuracy in meters, then there is a 68\% probability that the true
  location is inside the circle.

\item Geometric Dilution of precision (\textbf{GDOP}). It is a dimension less
  attribute and it is little bit difficult to understand this attribute. But in
  general it shows the effect of position of satellites in orbit on the GPS
  accuracy number that we get. A general rule of thumb is that a lower values
  (\textless 2)of dops gives high accuracy point. There are 3 DOP values that we
  get from cell phones:
  \begin{enumerate}
  \item HDOP - horizontal dilution of precision
  \item VDOP - vertical dilution of precision
  \item PDOP - position (3D) dilution of precision
  \end{enumerate}

  HDOP is related to the horizontal accuracy which is our lat-longs and VDOP is
  related to the accuracy of altitude. 

  From our GPS analysis we found that lower values of DOPs results in more
  accurate location point.

\end{enumerate}

\include{newresearch/newresearch}
\include{conclusion/conclusion}

\end{document}

