\documentclass[12pt]{report}
\usepackage{longtable}
\usepackage{tabu}
% \usepackage{pifont}
\usepackage{natbib}
\usepackage{graphicx}
\usepackage{caption}
\usepackage{multirow}
\usepackage{amsmath}
\usepackage{subcaption}
\usepackage{url}
\usepackage{hyperref}
% used for \textquotesingle
\usepackage{textcomp}
% for subfigure
% \usepackage[caption=false]{subfig}
% used for degree symbol
\usepackage{gensymb}
% \renewcommand\thesection{\arabic{section}}
% \renewcommand{\bibname}{References}

\usepackage[english]{babel}

\begin{document}

\begin{center}
  {\LARGE \textbf{GPS analysis for SafeStreet}}\\[0.3in]
  {\large \textbf{M.Tech Thesis Report}}\\[0.1in]
  {\large \textit{Submitted in partial fulfillment of the requirements}\\
    for the degree of}\\[0.1in]
  {\large \textbf{Master of Technology\\
      in\\
      Computer Science and Engineering}}\\[0.2in]
  \large by\\[0.1in]
  {\Large Vikrant\\Roll No. 153050031}\\[0.1in]

  \Large \textit{under the guidance of}\\[0.1in]
  {\Large Prof. Bhaskaran Raman}\\[0.08in]
  \begin{figure}[h]
    \centering
    \includegraphics[width=2in]{iitb-black}
  \end{figure}

  {\Large Department of Computer Science and Engineering\\[0.08in]
    Indian Institute of Technology Bombay\\[0.08in]
    October, 2016 }
\end{center}
\pagebreak

\begin{abstract}
  SafeStreet is a pothole detection Android app \cite[]{safestreet_app}
  developed in IIT Bombay with a motivation to help the government know about the
  present road condition. Knowing the location of potholes can help the municipal
  corporation to repair them faster thereby increasing the vehicle's life and
  making the rider's journey pleasant.

  % As of now it is in priliminary stage of development.
  SafeStreet uses 3-axis accelerometer to detect the potholes and GPS to report
  it's location. SafeStreet version1 developed by Anupreet and Bhushan has been
  tested by 10 odd taxi drivers with a total distance of around 5000 Km,
  reporting around 10000 potholes/bumps \cite{anupreet} \cite[]{bhushan}. After
  analyzing the data collected by SafeStreet v1 \cite{anupreet}, I found that it has
  56772 pothole reports with an average distance error of 4.3 meters across all
  vehicle types from the nearest road and 3956 reports have distance from them
  nearest road greater than 10 meters with a maximum error distance of
  \textbf{600 meters}. The reason for this large error distance is location
  updates received from cellular towers which can have accuracy of 100's of
  meters, if we stop listening location from cellular network then this 600
  meters of distance error can be brought down to 38.6 meters
  
  This report presents the GPS outage and accuracy analysis which has helped in
  answering the question as to why there is so much distance error with GPS. GPS
  accuracy analysis has been done for GPS only, A-GPS (Assisted) and A-GPS with
  network settings with different vehicle types. Based on the results from GPS
  analysis I have changed the location listener API from Google
  \cite[]{google_location_api} to Android \cite[]{android_location_api}. Apart
  from this, design changes has been made to make the SafeStreet more robust
  and scalable.


\end{abstract}

% {{{ ack Page number at bottom
\thispagestyle{plain}
% Title
\begin{center}{\huge{\textsc{Acknowledgements}} \par}\end{center}
\vspace*{15px}
% Write acknowledgement here
\noindent

I would like to thank my guide, Prof. Bhaskaran Raman for his constant guidance.
His plan of conducting two meetings in a week proved to be an effective idea
which helped in quickly correcting mistakes and minimized the effort in the wrong
direction. I thank him for giving me his scooter \textit{n} number of times at
any time in day or night.

I would also like to thank Nivia Jatain for helping me in data collection
process. She also helped me in proofreading this report.

I would also like to thank Deepthi for helping me in carrying out the project
smoothly.


\vspace*{15px}
\begin{flushright} {\textbf{Vikrant}}
\end{flushright}
% }}}


\pagenumbering{roman}
\tableofcontents

\setcounter{page}{1} \pagenumbering{arabic}

\chapter{Introduction}
\label{sec:introduction_sec}
% DONE: What is the setting of the problem? DONE: Why is the problem important
% to solve? This is the motivation.
India had a road network of 46.9 lakh kilometers (Approx) in 2011 \cite[]{nhdp},
which is $2^{nd}$ \cite[]{cia}largest road network in the world. For road maintenance
$13^{th}$ Finance Commission has granted Rs 199,300 million \cite[]{rural_roads}
over a period of 4 years (2011-2015). Knowing the location of potholes helps
municipal corporation to know which road needs fixing thereby making the
road repair process faster. SafeStreet is a crowd source solution to address
this problem by marking the location of potholes accurately on the Google maps.

What is SafeStreet ? SafeStreet is an automated pothole detection and reporting
Android application\cite[]{safestreet_app}.
\begin{figure}[h]
  \centering
  \includegraphics[width=2in]{images/phone}
  \caption{Showing phone's positive x,y,z coordinate axes. positive z-axis is in
    the direction of phone's screen facing the sky. }
  \label{phone_axes}
\end{figure}

It uses 3-axis phone's accelerometer sensor to detect potholes and braking
events. Potholes are detected by monitoring the acceleration in \textit{z}
direction where $\sqrt{a_{x}^{2} + a_{y}^{2}}$ helps in determining the braking
event. Accelerometer data is then sent to the server along with GPS data to mark the
location of potholes on Google maps. Compared to threshold based methods
\cite[]{mohan2008nericell} SafeStreet uses machine learning technique to detect
the potholes and braking events.

SafeStreet is a scalable solution for road
monitoring as it doesn't require any special hardware installation on the
vehicles like others \cite[]{eriksson2008pothole} have used.

Virtual reorientation algorithm \cite[]{bhushan} \cite[]{auro} takes care of
restriction of keeping the phone in some particular orientation e.g. screen sky
facing to measure the acceleration in \textit{z}-direction. Now user can keep
his phone anywhere or in any orientation while riding. For best results phone
position should be kept in the same orientation throughout the ride.


% SafeStreet version 1 has been tested for \textgreater 5000 kilometers by 10
% taxi drivers \cite[]{anupreet}. 56,772 pothole reports from different vehicle
% type (auto,bus,car and bike) has been sent to server till 22nd July 2016 with
% an average distance error of 4.3 meters across all vehicle types from the
% nearest road and 3956 pothole reports have distance from nearest road greater
% than 10 meters. A maximum error distance of \textbf{600 meters} has been
% observed. The reason for this large error distance is location received from
% cellular towers which can have accuracy of 100's of meters. From our GPS
% analysis it is observed that 600 meters of distance error has brought down to
% 38.6 meters if we stop
% listening for location updates from cellular network.\\

\textbf{Problems in SafeStreet}: Following are the shortcomings in the old
app/server:

\begin{enumerate}
\item \textbf{GPS Error}:

  3956 out of 56772 pothole reports are off by greater than 10 meters from the
  nearest road. A maximum distance error of 600 meters from the nearest road has
  been found and this is not acceptable for any pothole detection system. The
  location obtained from GPS does not follow a straight line even if the vehicle
  is moving in a straight line. Sometimes points may be marked on the opposite
  road and this will lead to false pothole detection. Figure \ref{gps_jittering}
  illustrates this.

  \textbf{Solution:} By using Android location API for location listener we can
  make our app not to look for location updates from cellular networks which has
  an accuracy of 100's of meters. In Google location API, if there is no GPS fix
  then this API determine the phone's location from cell towers which have
  accuracy in ranges of 100's of meters.

\begin{figure}[h]
    \centering
    \includegraphics[width=3.5in]{images/gps_jittering}
    \caption{GPS error illustration}
    \label{gps_jittering}
  \end{figure}


\item \textbf{Age of GPS location was not considered in labeling potholes with
    lat-long}: Whenever a pothole is detected by the app, the app doesn't check
  how old is the last known GPS location. If last known location is older than 2
  seconds from the pothole detection time then we should ignore it as it would
  add an error equal to the speed of the vehicle (in m/s) times the age of last
  known GPS location (in seconds).
  
  \textbf{Solution}: A simple solution is, instead of ignoring old GPS points
  (time \textgreater \ 2 seconds), store both pothole detection time as well as
  GPS timestamp which will help in filtering out pothole location whose
  difference in pothole detection timestamp and GPS timestamp is greater than 2
  seconds. Compared to ignoring, storing the timestamp data of pothole and GPS
  has an advantage of interpolating the GPS location for potholes whose
  associated GPS location age is greater than 2 seconds.

\item \textbf{Location updates from cell towers are not filtered}:
  % If we imagine that our phone is not receiving location updates from GPS and
  % we
  % have configured the app to receive location from both GPS and network
  % provider
  % then we get pretty bad accuracy for network points.
  The location updates from cell towers have low accuracy (in the range of 100s of
  meters). As we can see in figure \ref{network_vs_gps} network points (the blue
  one) are off the road by a big margin whereas GPS points (the green one) are
  tracing the road pretty closely.
  \begin{figure}[h]
    \centering
    \hspace{.2in} \includegraphics[width=5.1in]{images/network_vs_gps}
    \caption{Showing location points received from cellular towers (Blue mark)
      vs GPS(Green mark). Starting from Kresit to Hostel 14 inside IIT Bombay}
    \label{network_vs_gps}
  \end{figure}

  % \item \textbf{Posting of junk data by malicious users is not handled}:
  %   Anyone can post any data given that he/she knows our API url. This will
  %   add erroneous data to the database. Solution for this has been discussed
  %   in section \ref{sec:app_side}

\item GPS outage (more details in sec \ref{sec:gps_outage}) was not taken into
  consideration.

\item \textbf{Not sending raw accelerometer data to the server}: The old version
  of app, re-orient the accelerometer data and send it to server for further
  processing, this is not a good idea because if in future there is a need to
  modify re-orientation algorithm then the whole process of data collection will
  have to be carried out again.

\end{enumerate}

All these shortcomings led to a new design which handles the above mentioned
problems.

% Is the problem still unsolved? The constitutes the statement of past/related
% work crisply. Related work


% I have done the GPS analysis and SafeStreet code review to figure out the
% cause of this high error.

% Based on GPS anaylsis code review, I have changed the location listener in
% SafeStreet from Google location API \cite[]{google_location_api} to Android
% location API \cite[]{android_location_api}.

% Because of the nature of data that we collect has lot of redundancy, we have
% used data compression technique to compress the data before sending it to
% server. This will save 60-70\% of interent usage.
  

% the location listener from Google location API \cite[]{google_location_api} to
% Android location API \cite[]{android_location_api}.

% analysed their pothole report data as well as their app/server code. We found
% that most of have come up with new app/server design to improve the location
% accuracy of detected potholes and for this we have done GPS analysis and made
% some design changes in SafeStreet v1. Unlike installing separate GPS and
% accelerometer sensors on vehicle \cite[]{pothole_patrol} we use

% What exactly is the problem you are trying to solve? This is the problem
% statement.
\noindent
Problem Statement: This report addresses the following questions.

\begin{enumerate}
\item \label{q1} What is the percentage GPS outage for different vehicle types
  (car, bus, scooter/cycle and train)?

\item What is percentage GPS outage when operating phone on different GPS
  settings?

  These settings are:
  \begin{enumerate}
  \item \textbf{GPS only mode} : In this mode, app is configured in such a way
    that it doesn't receive location updates from cell towers. Also the internet
    connection is turned off.
  \item \textbf{Assisted GPS (A-GPS)} : Same settings as in point (a) but
    internet connection is turned on.
  \item \textbf{A-GPS with network} : Same setting as point (b) but app is
    configured to receive location updates from cell towers.
  \end{enumerate}

\item What accuracy numbers do we get when we operate Android phones on
  different GPS settings for different vehicle types?
\item Is there any effect of vehicle's speed on GPS accuracy numbers?
\end{enumerate}

% Why is the problem difficult to solve? This solution from this problem that
% still needs to be solved 1. high battery usage when GPS is turned on. 2.
% Handling GPS outage. There are cases when phones doesn't get any GPS fix. GPS
% outage depends on phones and vehicle type.

% How have you solved the problem?


% What are the conditions under which your solution is applicable? This is a
% statement of assumptions.

% What are the main results? You have to present the main summary of the results
% here.


% What is the summary of your contributions?


% How is the rest of the report organized?
\section{Outline}
This report is organized as follows,
Chapter 2 discusses the related work,
Chapter 3 describes the logistics used for experimentation,
Chapter 4 presents the GPS analysis,
Chapter 5 describes the experiments conducted to validate the distances measured
on Google maps, 
Chapter 6 discusses the new changes in SafeStreet app/server,
Chapter 7 presents challenges faced while developing the app,
Chapter 8 discusses future work and Chapter 9 concludes the report.


% \chapter{Motivation}
% \label{sec:motivation_sec}

% \section{GPS Analysis}
\subsection{How does GPS works ?}
Before we dig deep into the analysis GPS thing lets first understand how does GPS works?

As we know that GPS works on the principle of trilateration which means that if we
want to find out the location of a point in space and we know that it lies in three
planes then we can uniquely find out this point. Yes you guessed it right it the
point of intersection of all three planes. But in practice 

\begin{figure}
  \centering
  \begin{minipage}{.5\textwidth}
    \centering
    \includegraphics[width=.4\linewidth]{analysis/satellite}
    \captionof{A cell phone receiving signal from satellites in sky}
    
  \end{minipage}%
\end{figure}
\begin{figure}[h]
  \centering
  \includegraphics[width=2in]{analysis/trilateration}
  \caption{Trilateration}
  % \ref{trilateration}
\end{figure}



% Because of all these shortcomings we have come up with a new design changes
% which will try to solve the above mentioned problems.
\chapter{Logistics used in experiments}

% In this report we have done the analysis of GPS data obtained in our data
% collection phase.

% Before I answer this question lets first see the experimental setup.
To determine the \% GPS outage, comparing GPS accuracy numbers and finding the
effect of vehicle speed on GPS accuracy with different GPS settings (GPS only,
A-GPS , A-GPS+internet) for different vehicle types (scooter, auto, car and bus)
I have made a total of 32 trips between

\begin{enumerate}
\item Hostel 14 and Kresit
\item Kresit and Ghatkopar RCity Mall and
\item Kresit and Kasarvadavali (Thane west)
\end{enumerate}

For experimentation, I have used my cycle, Suzuki Access 125 scooter,
Autorickshaws, Maruti SX4 (car), IITB TumTum, Mumbai roadways buses and Mumbai
local (2nd class).

The phones which were used for data collection includes Oneplus One, OnePlus X,
Samsung Grand Max, Samsung Grand Prime, Samsung Galaxy S3, Samsung S3 Neo, LG
Nexus 4, LG Nexus 5X, LG Nexus 5, Moto-E and HTC Desire 626 g+.

I have used GPS logger app \cite[]{gps_logger_app} \cite[]{gps_logger_website}
to collect GPS data.

 \begin{figure}[h]
   % \centering
   % \begin{subfigure}[t]{.5\textwidth}
   %   \centering
   %   \includegraphics[width=2.3in]{images/logger_app}
   %   \caption{GPS logger app by me}
   % \end{subfigure}%
   % ~
   % \begin{subfigure}[t]{0.5\textwidth}
     \centering
     \includegraphics[width=2.3in]{images/gpslogger_app}
     \caption{Open sourced GPS logger app used for data collection}
   % \end{subfigure}
   % \caption{GPS logger apps used in data collection}
   \label{gpslogger_app}
 \end{figure}

 \pagebreak
 % I have developed a GPS logger app which receives location updates
 % from both Google Location API as well as Android Location API so that we can
 % compare their accuracy numbers. But, we later found that Google API is not
 % suited for our application because of the reasons discussed in section
 % \ref{conclusion} therefore we stopped using Google API.

 % The reason why

 % I have built a separate app for GPS logging that logs all the GPS details which
 % GPS logger app \cite[]{gps_logger_app} \cite[]{gps_logger_website} logs in
 % addition my app also logs the vehicle type information used for ride. For this
 % a dropdown menu is incorporated on the top of the screen to select the vehicle
 % type. One text box is also made to enter any remark like whether condition and
 % current GPS settings of phone during ride.

 % While developing this app, I figured out that there is no direct API for
 % getting the number of satellites used in GPS fix and also for \textit{DOP}
 % related information we had to separately parse the NMEA string \cite{nmea}.

%  \chapter{GPS Analysis}
%  \section{How GPS works ?}
%  Before we dig deep into the GPS analysis let's first understand how GPS works?

% \begin{figure}[h]
%   \centering
%   \includegraphics[width=2.5in]{analysis/satellite}
%   \caption{A cell phone receiving signals from 4 satellites}
%   \label{satellite}
% \end{figure}
% As we know that GPS works on the principle of trilateration which means that we
% can uniquely identify the location of a point in space if we know
% it\textquotesingle s distance from three points. In our case these 3 points are
% satellites and cell phone is the point which we want to identify uniquely.
% \begin{figure}[h]
%   \centering
%   \includegraphics[width=2.5in]{analysis/trilateration}
%   \caption{Trilateration}
%   \label{trilateration}
% \end{figure}

% GPS receivers (in cell phones) use accurate time difference to calculate their
% distance from satellites, which is equal to the difference between the arrival
% time of GPS signal and the time at which signal was sent. But it is not that
% simple to get the time in nanosecond range with our cell phones as it doesn't
% have atomic clock unlike satellites. The reason we need nanosecond accuracy in
% time difference is that an error of just 30 nanoseconds in time can lead to an
% error of 9 meters on Earth which is large enough to change pothole location from
% one lane to another. To solve the need of high accuracy clocks in our cell
% phones, cell phones uses 4 satellites.

% \newpage

 \chapter{GPS analysis}
\section{GPS outage comparison for different vehicle type and different GPS settings}
\label{sec:gps_outage}
\% GPS outage is defined as :
\begin{equation}\label{gps_outage_eqn}
  \%GPS\  outage = 100 \  - \  \frac{number\  of\  GPS\  location\  updates\  received}{total\  trip\  time\  in\  seconds}\  *\  100
\end{equation}

\noindent
In equation, \ref{gps_outage_eqn} \textit{total trip time} is the time when we
get the first GPS fix to the time at which we stop our GPS logging application.
Equation \ref{gps_outage_eqn} is valid only when our phone is set to receive
location updates after every 1 second. If the location update frequency is 1
after every $x$ seconds then the total trip time should be divided by $x$.
\\

Table \ref{gps_outage_table} presents the summary of \% GPS outage for different
vehicle types and GPS settings. Table \ref{gps_outage_table} shows that on an
average the GPS outage in buses and trains is around 33\% and 67\% respectively
whereas it is around 6-23\% in scooter, auto and car.


 \begin{table}[h!]
   \centering
   \caption{Percentage GPS outage for different vehicle type and GPS settings}
   \label{gps_outage_table}
   \begin{tabular}{|m{0.9in}|c|c|c|l|}
     \hline
     & GPS only & A-GPS & A-GPS + network  \\\hline 
     cycle/scooter                   & 5.44     & 7.98         & 56.59      \\ \hline
     auto                            & 14.62    & 0.00         & 3.03       \\ \hline
     car                             & 2.43     & no data      & 33.08      \\ \hline
     bus                             & 32.75    & no data      & no data    \\ \hline
     train                           & 91.03    & no data      & 43.79      \\ \hline
     average across all vehicle type & 29.26    & 3.99         & 34.12      \\ \hline 
   \end{tabular}
 \end{table}

 The observation that can be drawn from the above table is that it is difficult
 for cell phones to get a GPS fix in more crowded vehicles like bus and train in
 comparison to less crowded vehicles (like scooter, auto and car).

 % Question: What is percentage GPS Outage when operating phone on different GPS
 % settings?


 % As it is evident from table \ref{gps_outage_table}that there is no
 % correlation between
 
 As it can be observed from table \ref{gps_outage_table} percentage GPS outage
 is high for A-GPS and A-GPS+network settings in comparison to GPS
 only mode in the case of scooter/cycle and car whereas it is low for auto and
 train.

 When average percentage GPS outage is taken for all vehicles, A-GPS settings
 seem to have lowest whereas A-GPS+network have highest percentage outage. We
 have also observed that on few occasions, phones from different manufacturers
 have different GPS outage, traveling under the same condition and GPS settings
 figure \ref{gps_outage_nexus4} shows this behaviour.

 



 % , if we use GPS in combination with internet then GPS outage decreases for
 % auto and train but increase for scooter and car.
 % Although there can be many reasons for this behaviour, few of them are:
 % \begin{enumerate}
 % \item \textbf{Different phone hardware}: We have observed that some phones
 %   consistently have higher GPS outage compared to other phones which indicates
 %   that GPS accuracy is affected by phone\textquotesingle{}s hardware. Figure
 %   \ref{gps_outage_fig} shows how different phones behaves differently under
 %   identical conditions. If you look carefully, figure \ref{gps_outage_gp} shows
 %   that Samsung Grand Prime doesn\textquotesingle{}t get any location update in
 %   a run of 1.06 km.
 % \item \textbf{Different travel routes}: Trips done on cycle, scooter and auto
 %   were all within IIT campus whereas trips done on car, bus and train were
 %   outside of IIT campus. It was observed that we get better average accuracy
 %   numbers (\textless \ 5 meters) outside campus as compared to inside IIT
 %   campus for the same device. This behaviour can be because of large number of
 %   trees inside IIT campus which causes hindrance for GPS signal.

 % \end{enumerate}
 % phone hardware is one of factor affecting GPS accuracy.
 \newpage

 \begin{figure}[h!]
   \centering
   \begin{subfigure}[t]{0.33\textwidth}
     \centering
     \includegraphics[width=1.3in]{images/gps_outage_thane_west_to_thane_stn_nexus4}
     \caption{LG Nexus 4}
     \label{gps_outage_nexus4}
   \end{subfigure}%
   ~
   \begin{subfigure}[t]{0.33\textwidth}
     \centering
     \includegraphics[width=1.3in]{images/gps_outage_grandprime_kasarvadli_to_thane_station_bus}
     \caption{Samsung Grand Prime}
     \label{gps_outage_gp}
   \end{subfigure}%
   ~
   \begin{subfigure}[t]{0.33\textwidth}
     \centering
     \includegraphics[width=1.3in]{images/gps_outage_oneplusone_kasarvadli_to_thane_station_bus}
     \caption{Oneplus One}
     \label{gps_outage_oneplusone}
   \end{subfigure}
   \caption{Showing GPS outage for different phones for the same route}
   \label{gps_outage_fig}
 \end{figure}

 \newpage
 \section{Comparison of GPS accuracy number for different vehicle types and GPS
   settings}

 \label{gps_accuracy_for_different_gps_settings_and_vehicle_types}

 \textbf{GPS accuracy} is a number obtained from GPS device and it is measured
 in meters. Definition of accuracy according to \textit{developer.android.com}
 \cite{android} website is ``the radius of \textbf{68\%} confidence''. In other
 words, if you draw a circle centered at this location's latitude and longitude
 with a radius equal to the accuracy in meters, then there is a 68\% probability
 that the true location is inside the circle.

 \begin{table}[h]

   \centering
   \caption{Median accuracy (in meters) for different vehicle types and GPS
     settings}
   \label{median_accuracy_table}
   \begin{tabular}{|l|c|c|c|}
     \hline
     vehicle type/GPS settings & GPS only & A-GPS & A-GPS+network \\ \hline
     scooter/cycle                         & 12.3     & no data      & 9.9                  \\ \hline
     auto                                  & 15.9     & 5.0          & 6.2                  \\ \hline
     car                                   & 13.6     & no data      & 165.4                \\ \hline
     bus                                   & 21.6     & 17.0         & 25.4                 \\ \hline
     train                                 & 24.7     & no data      & 12.0                 \\ \hline
     average across all vehicles           & 17.6     & 11.0         & 45.8                 \\ \hline
   \end{tabular}

 \end{table}

\begin{table}[h]
  \centering
  \caption{Mean accuracy (in meters) for different vehicle types and GPS
    settings}
  \label{mean_accuracy_table}
  \begin{tabular}{|l|l|l|l|}
    \hline
    vehicle type/GPS settings   & GPS only & A-GPS & A-GPS+network \\ \hline
    scooter/cycle               & 13.8     & no data      & 49.4                 \\ \hline
    auto                        & 16.8     & 5.3          & 20.8                 \\ \hline
    car                         & 14.0     & no data      & 281.4                \\ \hline
    bus                         & 23.3     & 23.6         & 39.5                 \\ \hline
    train                       & 37.1     & no data      & 186.8                \\ \hline
    average across all vehicles & 21.0     & 14.4         & 115.6 \\  \hline
  \end{tabular}
\end{table}

 
 \begin{table}[h]
   \centering
   \caption{Percentage GPS accuracy less than 10 meters}
   \label{gps_accuracy_table}
   \begin{tabular}{|l|l|l|l|}
     \hline
     vehicle type/GPS settings   & GPS only & A-GPS        & A-GPS+network \\\hline
     scooter/cycle               & 40.6     & not sufficient data & 32.8                 \\\hline
     auto                        & 66.4     & 94.1                & 85.3                 \\\hline
     car                         & 68.1     & not sufficient data & 16.3                 \\\hline
     bus                         & 39.4     & 17.3                & 6.8                  \\\hline
     train                       & 20.5     & not sufficient data & 34.5                 \\\hline
     average across all vehicles & 47.0     & 55.7                & 35.1 
     \\ \hline
   \end{tabular}
 \end{table}
 \pagebreak

 % One conclusion that we can draw from

 Table \ref{median_accuracy_table} shows that average GPS accuracy for GPS only
 mode, A-GPS mode and A-GPS with network mode have 17.6, 11 and 45.8 meters of
 accuracy across all vehicles.

 On an average GPS only and A-GPS gives better accuracy compared to
 A-GPS+network mode. Hence it is good to ignore location updates from cell
 towers. This has become a motivating factor in switching from Google location
 API \cite[]{google_location_api} to Android Location API
 \cite[]{android_location_api}. The reason for switching to Android location API
 is that we can choose the source of location provider in Android location API
 whereas it is not possible to do with Google location API.

 % because if we use Google Location
 % API then we have no control to filter out the location updates from network.

 From tables \ref{median_accuracy_table},\ref{mean_accuracy_table},
 \ref{gps_accuracy_table} we can see that, GPS accuracy for scooter, auto and
 car is better than bus or train.

 % a good reason to justify this behaviour is
 % signal interference because of large number of nearby phones in crowded
 % vehicles.

 % which gives us the flexibility to not receive location
 % updates from cell towers whereas this is not possible to do it in Google API.

 % Only reason for using network as a source of location update is to get
 % faster location update which comes at the cost of lower accuracy.
 % \newpage
 
 % \section{Correlation between number of satellites used in GPS fix and GPS
 % accuracy number}
 % On an average GPS accuracy increases with the increase in number of
 % satellites used for GPS fix. But this is not always true, better accuracy can
 % be obtained with less number of satellites as shown in figure
 % \ref{accuracy_vs_satellites_scatter_plot}.
 % \begin{figure}[h]
 %   \centering
 %   \begin{subfigure}[t]{0.5\textwidth}
 %     \centering
 %     \includegraphics[width=3.2in]{images/accuracy_vs_satellites}
 %     \caption{Average accuracy vs number of satellites used in GPS fix}
 %   \end{subfigure}%
 %   ~
 %   \begin{subfigure}[t]{0.5\textwidth}
 %     \centering
 %     \includegraphics[width=3.2in]{images/accuracy_vs_satellites_scatter_plot}
 %     \caption{Accuracy vs number of satellites used in GPS fix}
 %     \label{accuracy_vs_satellites_scatter_plot}
 %   \end{subfigure}
 %   \caption{Accuracy vs number of satellites}
 %   \label{accuracy_vs_satellites}
 % \end{figure}
 % \newpage
\section{Vehicle speed vs Accuracy}
\begin{figure}[h!]
  \centering
  \includegraphics[width=4in]{images/accuracy_vs_speed3}
  \caption{Average accuracy vs. Speed value received from GPS}
  \label{accuracy_vs_speed_label}
\end{figure}
Figure \ref{accuracy_vs_speed_label} shows that speed value returned from GPS doesn't
have any significant effect on GPS accuracy. There is only 5 meters of variation
observed in accuracy for all speed ranges.

\chapter{Experiments}
\section{Ground truth testing for GPS location}
\begin{enumerate}
\item \textbf{Experiment 1: Validating Google maps distances for inside IIT
    campus}: Lat-longs of potholes are just two numbers it doesn't have any value
  until it is plotted on some map. So it is important to validate the Google
  map as all the pothole locations are shown on the Google maps.

  In our ground truth testing experiments we found that distances
  measured on Google maps doesn't match with the real ground distances. To
  testify this, we measured the width of the K V school road inside IITB with a
  rope and it was found to be 8 meters whereas on Google maps it showed only 4
  meters. So we can conclude that Google maps doesn't always reflect the
  absolute ground truth, they just show an approximate measurement and therefore
  should not be used for accurate measurements purposes.

\item \textbf{Experiment 2: Finding the average and maximum distance of GPS
    points logged by the stationary phones from it's true location}: 

  This experiment shows the behaviour of the GPS points logged by the stationary
  phones which are kept on a known location (we know the lat-long of that point)
  so that later we can measure the distances of GPS points from its true
  location. For this, we have chosen 3 potholes near road intersection and kept
  4 phones in GPS and A-GPS mode for 1 minute turn by turn.

  We have chosen known pothole locations for this experiment so that later we
  can compare the position of this pothole reported by our app with what we
  observed during experiments. Although we haven't done this part yet.

  \textbf{Result} : The average error distance found for pothole\#1 from it's
  true location is \textbf{7.26 meters} and \textbf{76.11 \%} of times location
  logged by phones is within a circle of radius 10 meters from its true
  location. The farthest point was \textbf{25.8 meters} away.

  \textbf{Conclusion}: There is 76 \% probability that pothole detected by with
  the help of GPS will be within a circle of radius 10 meters from its true
  location.


  % as we can know the location of road intersections from the
  % Google maps. To know the exact location of these potholes we have
  % measured their distances by rope from the road intersection.

  % We have conducted the experiments to analyze how accurately we can find out
  % the location of potholes with GPS. For this we chose 3 potholes inside the
  % campus and kept 4 phones near the potholes for which we knew the exact
  % location and logged the GPS points for 1 minute in GPS only and A-GPS mode.

  \begin{figure}[h]
    \centering
    \includegraphics[width=4in]{images/pothole1}
    \caption{Showing the GPS points on Google map. GPS points are logged by 4
      different phones for 1 minute, Point P in red color denotes the true
      location of the phones. Number near markers denotes GPS accuracy
      of the points. Phone location is 44.5 meters away from the intersection of
      Brewberry road and Hostel 5 road. A thumbnail photo shows 4 phones lying
      pothole while logging the GPS data}
  \end{figure}



\end{enumerate}


\chapter{New changes in SafeStreet}

SafeStreet server as well as app code has been put under git version control and
is hosted on Github.

App URL is \url{https://github.com/vikrant1433/safestreet2-app/tree/develop} \\
Server URL is \url{https://github.com/vikrant1433/safestreet-server/tree/develop3}

A documentation generator tool called Sphinx \cite[]{sphinx} has been integrated
with SafeStreet which will use doc string comments from classes and function
definitions to generate documentation.

SafeStreet documentation has been put up on readthedocs
\url{http://safestreet-server.readthedocs.io/en/latest/} which automatically get
update on every code push to Github. But as there is no documentation written by
our seniors in the standard format, there is no documentation available for this
project. For future purposes this documentation system will be helpful.
 
\section{App Side}
\label{sec:app_side}
\begin{enumerate}
\item \textbf{Compressing log files before sending to server}: We have observed
  that there is a lot of data redundancy in log files because of nature of
  data that we collect (First 3 digits of latitude and longitude changes after
  traveling distance greater than 11 Km). After compressing, file sizes are
  reduced by 60-70\% which saves user's internet cost. We have tried many
  compression algorithms but zip4j is found to be a better choice for our
  application. Figure \ref{compression} shows the comparison of different
  compression algorithms ran on our log files.
  \begin{figure}[h]
    \centering
    \includegraphics[width=6in]{images/compression7}
    \caption{Comparison of different compression algorithm for file size of 173.9 MB}
    \label{compression}
  \end{figure}

  % \label{api_security}
% \item \textbf{Preventing upload of tampered files} : To prevent tampering of log
%   files, we have calculated the md5 checksum of log file's content + user's
%   email id which will be send with every file upload request to server, then on
%   server side we recompute the md5 hash of file content + user's email id and if
%   it matches then we are sure that uploaded files are not tampered and if it
%   doesn't we simply discard those files. The reason for including some
%   additional text with log file which is usually called as ``salt'' is to
%   prevent the malicious user from computing our secret checksum even if
%   he/she knows what hashing algorithm we have used.

%   This scheme is better compared to sending encrypted files and then decrypting
%   them on server side because encryption and decryption operations are slower than
%   computing md5 hash and moreover we want data integrity not confidentiality.

 
\item Using Android SQLite database with GreenDAO \cite{greendao} as an ORM
  (Object Relational Mapper): An ORM is an abstraction layer which helps us to
  interact with the database table in an object oriented manner.

  Writing plain old SQL queries very tedious and maintenance become painful as
  more features get added to the project.


  In ORM, database table becomes class, columns becomes the member variables and
  each row in the table is an object of that class.

  Using an ORM we are not bothered about writing SQL to create tables and
  managing column constraints that will be managed by ORM layer.

  In the old app, information about the upload status of log files is stored in
  files which is very inconvenient . It is difficult to keep
  track of which files are yet to be uploaded.

  Now the upload status of files is stored in the database which will be queried
  every hour and sent to the server if internet connection is on.

\item Use of new view library called ButterKnife \cite{butterknife} to reduce
  writing of redundant UI code. Using ButterKnife library helps in focusing more on
  business logic rather than worrying about UI code wiring. Less code means less
  maintenance.
  \begin{figure}[h!]
    \centering
    \begin{subfigure}[t]{0.5\textwidth}
      \centering
      \includegraphics[width=3in]{images/redundant_code3}
      \caption{Redundant code}
      \label{redundant_code_fig}
    \end{subfigure}%
    ~
    \begin{subfigure}[t]{0.5\textwidth}
      \centering
      \includegraphics[width=3in]{images/concise_code3}
      \caption{Concise code using ButterKnife}
      \label{concise_code_fig}
    \end{subfigure}
    \caption{Eliminating redundant UI code}
    \label{redundant_concise_fig}
  \end{figure}

\item \textbf{Exponential backoff retries requests in case of failed upload
    files}: In case there is a failure in posting data to the server which can
  be because of server overload or loss of internet connectivity, the app wait
  time before retry will increase in an exponential manner (1,2,4,8,16 seconds )
  before it reaches it's maximum retry limit (5 in our case), this will happen only
  if there is an internet connection and users has pending files to upload.
  After maximum retry limit has reached user can manually press upload button to
  upload the log files to server or app will automatically try to send the
  pending file every time the app gets restarted thereby required minimum user
  intervention.

\end{enumerate}

\section{Server Side}
\label{sec:server_side}

Server side technologies:
\begin{enumerate}
\item \textbf{Web framework}: We use open sourced Python-based Django
  web framework \cite[]{django} to process all our log file and displaying
  potholes markers on Google maps. All our functionalities are REST APIs which
  are build with \textit{django restframework} (\textit{drf}).
  \item \textbf{Webserver}: We have migrated from using django development server
    to Nginx with Gunicorn which is more powerful and flexible than Django's
    built-in server. Automatic restart script has been written to restart (nginx+gunicorn)
    servers on power failures.
  \item \textbf{Database}: We use Postgres database to store all ride details sent
    from Android devices.
\end{enumerate}

% , running the clustering algorithm

% Java implementation of reorientation algorithm has been removed from client
% side and is now implemented on server side.

% Mechanism to detect upload of tampered files In re-orienting the accelerometer
% data, passing it to svm predictor, saving pothole location in database,
% running the clustering algorithm and finally showing pothole on map.

% The advantange of this design is that now we don't need


% \section{Design comparison of SafeStreet versions (App side)}


% \section{Design comparison of SafeStreet versions (Server side)}

\section{Hurdles in implementation}
\begin{enumerate}
% \item \textbf{Inconsistent comments in code}: Comments written in old code
%   base is not consistent with what code do. As you can see in figure
%   \ref{inconsistent_comments_fig}, comment tells that we consider a range of
%   data of moving vehicle if speed is greater than 10 meters/second whereas they
%   mean to say 10 kmph.

  % Remember one thing converting comments to code is easier
  % than other way around so comments should written with at most care as it will
  % save hours of works for others and also help you understand the code faster.
  % \item \textbf{Showing multiple marks for single Pothole}:
  % \begin{figure}[h]
  %   \centering
  %   \includegraphics[width=5in]{images/inconsistent_comments}
  %   \caption{Misleading comments}
  %   \label{inconsistent_comments_fig}
  % \end{figure}
\item \textbf{Supporting an app to run on different versions of Android}:
  Because of rapid development in Android ecosystem, Google frequently releases
  the new versions of Android tools and libraries. To make our app run on most
  of the devices we have to update our tools and libraries too, which sometimes
  break the build process of our application and thus lowering the developer's
  productivity.
\item \textbf{Introduction of new permission system for Android Marshmallow and
    later versions }: Recently Google has introduced a new permission system for
  Marshmallow (API level 23) and above which asks for user permission at runtime
  rather than at app installation time. This new permission system has made the
  user's life easier by giving them the power to selectively grant or deny
  permission for the use of different system resource. But now developers have
  to check for permission every time before using system resources (GPS or
  accelerometer) and this has made the app updation difficult from Android
  Lollipop(API level 22) to Marshmallow. Now SafeStreeet v2 supports Android
  platform Marshmallow (API level 23) and above
\end{enumerate}
\chapter{Future Work}

\begin{enumerate}
% \item \textbf{Integrating clustering to eliminate duplicate potholes} : This is
%   the first thing that has to be done in MTP phase-2.

\item \textbf{Ground testing of SafeStreet v2}: As Phase-1 was mainly focused on
  anaylsis and implementing features in SafeStreet, we need to do the ground
  testing in Phase-2 to compare the performance with SafeStreet v1.

\item \textbf{Incentivize user to use our app by building an early warning
    system} : To build a more robust pothole detection app we need to collect
  more data. It is very likely that people won't use this app as it is very
  power demanding. So to make others use this app we can integrate an early
  warning system which will tell users about the upcoming pothole on their way
  and also recommend the route based on road condition.

\item \textbf{Automatic detection of repaired potholes} : There should be a
  mechanism by which we can tell if the potholes that are shown on the map is no
  longer valid. A simple solution for this problem is to mark potholes as
  repaired if we are not getting the report for the pothole for a long period of
  time say 15 days.


\item \textbf{Detecting uncalibrated or poor device sensors} : While developing
  SafeStreet app we haven't taken into the account the fact that not all device
  have good accelerometer sensor. Older devices may have poor sensors which may
  lead to false positive/negative pothole dectection. So identifying such
  devices can be a challenging task. One way to eliminate some calibration error
  is to ask users to keep their phone stationary for few minutes and then
  subtracting any offset error if present. Although in practice it would be a
  difficult task.

\item \textbf{Making app battery efficient} : We all know that higher the rate
  of GPS location update higher will be the battery drainage. Sadly our app has
  been configured to receive GPS update at the highest possible rate which will
  restrict our app usage for long period of time without a power bank.
 
\item \textbf{Doing the scalability testing of SafeStreet}: Now we don't know
  how many users can our SafeStreet server handle.

\end{enumerate}

% is that if we lower the location update rate then it will add more error to
% our detected pothole location.


\chapter{Conclusion}
\label{conclusion}
After doing analysis on GPS accuracy for different phones on different
  vehicles with different GPS settings it is found that GPS accuracy can be as
  good as your mobile hardware supports. It is the app developer's
  responsibility to filter out the bad data.


  % The implication of all this GPS analysis is two fold first is now we know
  % that how phones behave differently with different GPS settings.

  In our experiments, we have used phones from different manufacturers which
  add more non-uniformity in our analysis. It would be better if we could at
  least remove this non-uniformity by using the identical phones for GPS
  analysis.

  If you are developing an app which has stringent accuracy requirements then
  don't use Google Location API \cite[]{google_location_api} because then you
  have no control over from which location provider to receive location updates.

% With the help of GPS accuracy analysis we found that configuring application
% to receive location update from network is not desirable for our application.
% This made us to take the decison of changing location listener from Google
% Fused API to Android location listener which give us control of not receiving
% location updates from network.
 
\bibliographystyle{unsrt}
\bibliography{references}

\end{document}
